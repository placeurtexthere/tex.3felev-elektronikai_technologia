\documentclass[12pt]{article}
\usepackage{amssymb}
\usepackage{amsmath}
\usepackage[usenames,dvipsnames]{xcolor}
\usepackage{indentfirst}
%\usepackage{fullpage}
\usepackage[a4paper, total={170mm,257mm}, left=20mm, top=20mm]{geometry}
\usepackage[symbol]{footmisc}

\renewcommand{\thefootnote}{\fnsymbol{footnote}}

\setcounter{section}{-1}
	
\renewcommand*\contentsname{Tartalomjegyzék}

\AddToHook{cmd/section/before}{\clearpage}

\begin{document}
\begin{titlepage}
	\centering \vfill
	{\textsc{Budapesti Műszaki és Gazdaságtudományi Egyetem} \par} \vspace{7cm}
	{\huge\bfseries A számítástudomány alapjai\par} \vspace{0.5cm}
	{\large \textsc{Összefoglaló jegyzet}\par} \vspace{1.5cm}
	{\Large\itshape Készítette: Illyés Dávid\par} \vfill

	\noindent\fbox{%
    	\parbox{140mm}{
			\color{red}\textbf{Ez  a jegyzet nagyon hasonlóan van struktúrálva az előadás jegyzetekhez és fő célja, hogy olyan módon adja át a "A Programozás Alapjai 1" nevű tárgy anyagát, hogy az teljesen kezdők számára is könnyen megérthető és megtanulható legyen. }
   		}
	}
	
	\vfill {\large \today\par}
\end{titlepage} 
\tableofcontents
\addtocontents{toc}{~\hfill\textbf{Oldal}\par}

    \section{0 Bevezetés, fogalmi rendszerezés}

        \paragraph{Mi az Elektronikai Technológia?} A technológia az anyag jellemzőinek tervezett, maradandó megváltoztatása. Az elektronikai texhnológia a villamosmérnöki tudományos és Ipari-kereskedelemi ismereteknek azon területe, amely az elektronikus áramköri egységek alkatrészeinek, hordotóinak és összeköttetés rendszereinek tervezésével, megvalósításával és megbízhatóságával foglalkozik.

		\paragraph{Az elektronikai technológia hatóereje.} A funkciók integrációja a méret, az energiafelhasználás, a költségek és a környezeti terhelés optimalizálása, tervezhető megbízhatóság mellett.

		\paragraph{Mi az anyagismeret célja?}

			\begin{itemize}
				\item Az ipar különböző területein alkalmazható anyagok (természetes és szintetikus polimerek, fémek és ötvözeteik, egykristályos, kerámikus anyagok és kompozitok) felépítésének, fizikai, technológiai és használati jellemzőinek rendszerezése.
				\item Az anyagkiválasztás szempontrendszerének és módszertanának összefoglalása.
			\end{itemize}

		\paragraph{Mivel foglalkozik a tárgy?} 

			\begin{itemize}
				\item Elektronikus készülékek konstrukciós alapelvei, megbízhatóság és termikus tervezés.
			\end{itemize}

	\section{1 Elektronikus készülékek}

		\subsection{Készülékek fejlesztési fázisai}

			\begin{enumerate}
				\item Műszaki specifikáció meghatározása (50\%\footnote[1]{:a termék sikerességében való szerep aránya}):
					\subitem Egyeztetés, marketing, bench-marking, meglévő és várható előírások, hatósági előírások.
				\item Prototípus kifejlesztése (30\%\footnotemark[1]):
					\subitem Specifikáció, tesztelés, gyárthatósag, ár.
				\item Gyártástechnológia kidolozása (10\%\footnotemark[1])
					\subitem Gyártási költségek, gyártáskapacitás, tesztelés.
				\item Próbagyártás (10\%\footnotemark[1])
					\subitem Tesztelés (kihozatal/selejt arány).
				\item Gyártás (0\%\footnotemark[1])
					\subitem Minőségellenőrzés, SPC.
			\end{enumerate}

		\subsection{Út a műszaki specifikációig}
			
			\paragraph{Mit kell létrhozni?}

			A mérnöki gyakorlatban olyan készülékekkel foglalkozunk, amelyekre \underline{igény} mutatkozik.

			Az igény lehet:

			\begin{itemize}
				\item valós:
					\subitem egyedi (pl. atomerőmű),
					\subitem nem egyedi, vagy piaci (pl. autó), 
				\item látens (pl. SMS),
				\item a kitalálás pillanatában még nem létező (pl. Rubik Kocka)
			\end{itemize}















	\section{2 Elektronikai szerelési- és kötéstechnológiák}

			
\end{document}