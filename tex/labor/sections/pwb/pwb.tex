\documentclass[../labor.tex]{subfiles}


\begin{document}

    \subsection{A mérés célja} A nyomtatott huzalozások mechanikai, fotolitográfiai, valamint a szelektív rétegfelviteli és ábrakialakítási technológiáinak megismerése.

    \subsection{A mérési feladat} Olyan kétoldalas, furatfémezett, fényes ónbevonattal ellátott nyomtatott	huzalozású hordozó készítése, melynek előállításában a hallgatók is részt vesznek, és amelyet az Elektronikai Technológia gyakorlat más mérésein is fel fognak használni. A gyakorlatvezető irányításával a hallgatók elvégzik a nyomtatott huzalozású lemezek gyártásának főbb technológiai lépéseit.

    \subsection{A mérés elvégzésével megszerezhető képességek} Az alkalmazott berendezések működését és működési elvét, az alapvető technológiákat, a technológiák gyakorlati alkalmazását valamint a környezetvédelmi ismereteket tanulmányozzák a hallgatók.
    
    \subsection{A mérés során felmerülő fogalmak rövid meghatározása}

    \subsubsection{Nyomtatott huzalozású hordozó (lemez)} Az elektronikus készülékek, berendezések döntő többségénél a diszkrét aktív és passzív alkatrészeket nyomtatott huzalozású hordozókra szerelik. E hordozók feladata az alkatrészek mechanikai rögzítése és az alkatrészek kivezetői közötti villamos kapcsolat megteremtése szigetelő lemezen megvalósított vezető rajzolat révén.

    \subsubsection{Furatfémezés} A fúrás elvégzése, és megfelelő felületkezelés után a villamos szempontból szigetelő furatfalra árammentes, illetve elektrokémiai rétegépítési technológiákkal rezet, majd ónréteget visznek fel. A furatfémezés alkalmas vezető síkok közötti átvezetések (viák) kialakítására és a forrasztási felület megnövelése révén mechanikailag erősebb kötést kapunk. Már a huzalozástervezés során figyelembe kell venni a furatfémezés vastagságát a furatátmérő meghatározásakor. Fontos a furatfémezés minősége, elsősorban a megfelelő áramterhelhetőség, a jó tapadás és a forraszthatóság.

    \subsubsection{Rajzolatfinomság} A nyomtatott huzalozások fontos jellemzője a rajzolatfinomság, azaz a rajzolaton előforduló minimális vezeték- és szigetelő köz-szélesség. Ennek minimális értéke általában 0,1…0,3 mm, de ha az áramterhelhetőség, illetve az üzemi feszültség indokolja, akár néhány mm is lehet.

    \subsubsection{Fotolitográfia (fotoreziszt technológia)} A fotolitográfia során a hordozó felületére fényérzékeny anyagot, ún. fotorezisztet visznek fel. A fotoreziszt anyaga laminálással felvitt fólia (szilárd rezisztek), szitanyomtatással felvitt kétkomponensű lakk, vagy ún. függönyöntéssel felvitt folyékony reziszt. A függönyöntésnél megfelelő résen át szivattyú segítségével függőleges folyékony reziszt filmet hoznak létre, melyre merőlegesen halad a hordozó. A fotoreziszt film egyenletesen ráterül a hordozó felső oldalára. A folyékony reziszteket megvilágítás előtt beszárítják.

    Megvilágításkor az UV fény hatására kémiai kötések alakulnak ki, vagy bomlanak fel, miáltal megváltozik az előhívószerrel szembeni oldhatóság. A fotorezisztet ún. gyártófilmen keresztül világítjuk meg, mely a megvalósítandó rajzolatot tartalmazza. Az ún. negatív működésű reziszteknél a megvilágított részeken oldhatatlanná válik a fotoreziszt bevonat. Ritkán pozitív működésű reziszteket is alkalmaznak, amikor a megvilágított részek oldhatók az előhíváskor. Az előhívás után a felületen maradó szelektív fotoreziszt bevonat a maszk. Negatív maszkról akkor beszélünk, amikor a megvalósítani kívánt ábra negatívja a maszkkal fedett rész. Pozitív maszknál a maszk a rajzolatnak megfelelő részeket fedi. Előbbit szelektív rétegfelviteli (pl. galvanizálás), utóbbit réteg eltávolító technológiákhoz (maratás) használják. A gyakorlaton negatív maszkot készítünk.
    
    \subsubsection{Nedveskémiai bevonat-készítési technológiák} A nyomtatott huzalozások technológiájában a galvanizálást, az árammentes bevonat-felvitelt és az ún. immerziós bevonatkészítést soroljuk ide.

    Mindhárom eljárás folyékony közegben (elektrolitokban) kémiai reakciók révén megy végbe, ezért nevezzük e technológiákat nedveskémiai eljárásoknak. Közös tulajdonságuk, hogy e folyamatok mindegyike redukció: pozitív töltésű fém-ionok elektronfelvétellel fémmé redukálódnak.
    
    Galvanizálás során a redukció elektromos áram hatására megy végbe, árammentes rétegfelvitel esetén redukálószert használnak, az immerziós bevonat készítésekor pedig az elektródpotenciálok különbsége a folyamat hajtóereje.

    \subsubsection{Alkalmazott technológiák}
    \begin{itemize}
        \item[-] mechanikai technológiák: lemezollóval végzett darabolás, fúrás, csiszolás, kontúrmarás;
        \item[-] fotolitográfia: fotoreziszt felvitele, gyártófilm illesztése, megvilágítás, előhívás;
        \item[-] nedveskémiai technológiák: tisztítás, rétegfelvitel, rétegeltávolítás (maratás).
    \end{itemize}
        
        
        

    \subsubsection{A nyomtatott huzalozású hordozók gyakrabban használt alapanyagai} A különböző hordozók felhasználási területeit a vázanyag és a kötőanyag fizikai és kémiai tulajdonságai határozzák meg.
    
    Néhány elterjedten alkalmazott hordozó fontosabb tulajdonságai az 1. táblázatban láthatók.

    %táblázat


    %táblázat

    A megvalósítandó áramkörökhöz az optimális hordozó-anyagot az egyes tulajdonságok – pl. villamos és/vagy mechanikai tulajdonságok, környezetállóság, ár – mérlegelésével kell megválasztani. Olyan célra, ahol az ár a lényeges szempont, legtöbbször az olcsó papírváz erősítésű fenolgyanta lemezeket használják. Ezek hőállósága megfelelő, jól megmunkálhatók, de nagy a nedvszívó képességük és kicsi a mechanikai szilárdságuk. Gyártják önkioltó változatban is, ami azt jelenti, hogy a hordozó meggyulladása esetén égést elfojtó gázok keletkeznek Az ilyen tulajdonsággal rendelkező hordozók szabványos jelölése tartalmazza az FR (Flame Retardant) jelzést. A papírváz erősítésű epoxigyanta lemezek kis dielektromos veszteségi tényezővel és kedvező szigetelési tulajdonságokkal rendelkeznek. Jól sajtolhatók, hajlító szilárdságuk jobb, mint a fenolgyanta lemezeké. A megmunkálási körülményektől függően fémezett falú furatok készítésére is alkalmasak. Az üvegszövet erősítésű epoxigyanta lemezek kiváló villamos, mechanikai és hőállósági tulajdonságokkal rendelkeznek. Vízfelvételük csekély. Furatfémezési technológiákhoz kiválóan alkalmasak. A két oldalon huzalozott, fémezett furatú és többrétegű nyomtatott huzalozású lemezek legelterjedtebben használt szigetelőanyaga.
    
    Az említett hordozók mellett számos egyéb anyagot is használnak, melyek alkalmazását egy-egy különleges tulajdonságuk teszi indokolttá. Ilyen lehet például a kis dielektromos állandó vagy veszteségi tényező, a hőmérsékletállóság, hőtágulási és hővezetési tulajdonságok. Kiváló dielektromos tulajdonságai miatt a mikrohullámú elektronikában gyakran alkalmazzák a politetrafluoretilént (PTFE), közismert nevén a teflont. Alkalmazásának korlátja elsősorban az igen magas ár. A poliimidet jó szigetelési tulajdonságai mellett elsősorban az epoxigyantához képest magasabb hőállósága miatt célszerű esetenként használni.

    \subsection{A mérés menete}

    \subsubsection{CNC fúrás}

    A CNC fúráshoz köteget készítünk, azaz lemezollóval méretre vágjuk az FR4-es epoxi-üvegszövet hordozót, valamint a kifutólemezeket, majd két helyen fúrás után 3 mm átmérőjű csappal egymáshoz rögzítjük a lapokat. Fölső kifutólemezként 0,24 mm vastag kemény alumínium lemezt, alsó kifutólemezként pedig 2,5 mm vastag döntően műgyantából és farostból álló lemezt használunk. A termelékenység növelése érdekében rendszerint több hordozót fognak össze, egyszerre végezve el azok fúrását (1. ábra). A kötegben egyszerre fúrható hordozók számát (általában 2…4 db) a fúróprogramban előforduló legkisebb átmérőjű furatra számított furathosszfuratátmérő arány határozza meg, amelynek általában 7-nél nem szabad nagyobbnak lennie. A kifutó lemezek alkalmazásának célja elsősorban a sorjaképződés megakadályozása. Az alsó kifutólemez feladata egyben a munkaasztal védelme is. A kötegkészítést követő lépés a fúrás, ami CNC géppel történik (2. ábra).
    

    % kép

    % kép 

    A köteget a két csap alul kilógó részét megfogó pneumatikus szerkezet segítségével lehet a munkaasztalon rögzíteni. A fúróorsót ill. az asztalt nagy menetemelkedésű csavarorsók közbeiktatásával szervomotorok mozgatják a három tengely irányában. A gép működése közben keletkező fúrási törmeléket elszívó rendszer távolítja el.

    A számítógépes tervezés (CAD) befejezésekor gyártófájlokat készítenek, ezek egyike a fúrógépet vezérlő, rendszerint drl kiterjesztésű fájl, amelynek kiterjesztése a fúrás angol elnevezéséből (drill) származik. A fúróprogram tartalmazza azt az információt, mely meghatározza furatok átmérőjét és azok koordinátáit. A furatkoordináták átmérők szerint csoportosítva szerepelnek a fúróprogramban, az átmérő általában 0,1…6,3 mm közé esik, ennél nagyobb furatokat kör kontúrmarásával lehet kialakítani. 3 mm feletti átmérőjű furatoknál a légcsapágyazású orsók kímélése érdekében 1 mm körüli átmérőjű szerszámmal előfúrást végeznek, ezek a furatok a program elején helyezkednek el.
    
    A furatok elkészítésének sorrendje erőteljesen befolyásolja a termelékenységet, ezért a gyártás előkészítésekor az orsó útvonalát optimalizáljuk. Fúrás közben a számítógép kijelzi a gép működésének fő adatait: az éppen használt átmérőt, a fordulatszámot, az előtolási sebességet, az összes furat számát, az éppen készülő furat sorszámát és a fúró pillanatnyi koordinátáit. Az optimális technológiai paramétereket (fordulatszám, előtolási sebesség) az alábbi összefüggésekkel határozhatjuk meg:

    \[n = \frac{v}{d\pi} \]

    \[v_e = e \times n \]

    és ahol:

    \begin{itemize}
        \item{n: fordulatszám (ford/min)}
        \subitem{v: a fúró kerületi ("vágási") sebessége | 100$\dots$150 m/min,}
        \item{x d: a fúró átmérője | 0,1$dots$6,3 mm,}
        \item{$v_e$: előtolási sebesség (m/min),}
        \item{x e: előtolás | 0,05...0,15 mm/ford.}
    \end{itemize}

    \subsubsection{Nedves csiszolás}

    \subsubsection{Furatfémezés}
    
    \subsubsection{Fotoreziszt maszk készítés}
    
    \subsubsection{Rajzolatgalvanizálás}

    \subsubsection{Maszkeltávolítás és maratás}

    \subsubsection{Forrasztásgátló maszk és szelektív forrasztható bevonat felvitele}

    \subsubsection{Kontúrmarás}

    \subsubsection{Ellenőrző kérdések}






\end{document}