\documentclass[../labor.tex]{subfiles}


\begin{document}

        \setcounter{secnumdepth}{4}

        \paragraph{A mérés célja:}

            Olyan nyomtatott huzalozású lemez (NYHL, NYÁK) megtervezése, amely bevezet minket a CAD funkciók világába, és "hello world" jellegű alkalmazásként segít megérteni egy egyszerű felületszerelt áramkör születésének a menetét.

            A mérés során a hallgató megismerkedik az európai iparban jelentős SIEMENS Mentor Graphics PADS CAD szoftvercsaláddal és később a további mérések során 3D tervezésig, valamint alapvető szimulációkig jut a feladat. Az egyszerű kapcsolási rajzból kiindulva a hallgatóknak tehát a félév végéig egy egyszerű termék tervezésének a lépésein kell végigjutni, hogy az első ábrán látható eszköz tervéig eljussanak.

        \paragraph{A mérési feladat:}

            \sloppy A mérésvezető irányításával a hallgatók számítógépes áramkörtervező rendszerrel megtervezik az adott mintázattal rendelkező lemezt. A rajzolatot és a hozzá tartozó stencil apertúrákat úgy alakítják ki, hogy a lemez a legyártása után alkalmas legyen az adott stencil és NYHL technológia, valamint a szerelési-forrasztási folyamat határainak, tulajdonságainak megállapítására.

        \paragraph{A mérés elvégzésével megszerezhető késség    ek:}

            A hallgató megismerkedik a tipikus szereléstechnológiai kérdésekkel, a CAD szoftverek UX felületével, és általános felépítésével, munkafolyamatával (WorkFlow). A hallgató elsajátítja a mérés során használt áramkör tervező rendszer legfontosabb eszközeit. A hallgató nem lesz képes még önállóan ipari áramkörtervezői munkára, de a következő szintek elérését segíti a mostani gyakorlat, az általános "onboarding" a későbbi CAD munkák során könnyebbé válik, valamint a témába való bevezetés utáni fogalomrendszer-áttekintésre képes lesz.

    \subsection{Bevezetés}

        \subsubsection{Általános bevezetés, a Design Flow fogalma}

            \noindent Összességében a CAD-tervezés további ágait határozhatjuk meg:

            \begin{itemize}
                \item Mechanika
                \item Járműtervezés
                \item Lakás-, lakás- és építőmérnöki tervezés
                \item Kémia és molekuláris tervezés.
            \end{itemize}



\end{document}