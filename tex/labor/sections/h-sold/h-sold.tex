\documentclass[../labor.tex]{subfiles}


\begin{document}




    \subsection{A mérés célja}
		
    \subsection{A mérési feladat}

    \subsection{A mérés elvégzésével megszerezhető képességek}

    \subsection{A mérés során felmerülő fogalmak rövid meghatározása}

        \subsubsection{Forrasztás}

        \subsubsection{Javítás, rework}

        \subsubsection{Forraszötvözetek kézi forrasztáshoz}

        \subsubsection{Forraszhuzal}

        \subsubsection{Folyatószer kézi forrasztáshoz, rework-höz}

        \subsubsection{Forrasztási csomópont}

        \subsubsection{A forrasztás hőmérséklete}

        \subsubsection{Forrasztópáka}

    \subsection{A pákacsúcs hőmérséklete}

        \subsubsection{I. táblázat. Az áramkör alkatrészjegyzéke}

    \subsection{A mérés menete}

        \subsubsection{A munka megtervezése, alktrészek előkészítése}

        \subsubsection{Megfelelő pákacsőcs és pákahőmérséklet kiválasztása, beállítása}

        \subsubsection{A kézi forrasztás műveleti lépései}

            \setcounter{secnumdepth}{4}

                \paragraph{Felmelegedett forrasztópáka megtisztítása és előónozása}

                \paragraph{Hővezető híd képzése}

                \paragraph{A forrasztási csomópont kialakítása}

                \paragraph{Forraszhuzal elvétele, megfelelő intermetallikus réteg kialakítása (500 ms-1 sec), végül a pákacsúcs elvétele}

        \subsubsection{Felületszerelt alkatrészek beforrasztása}

        \subsubsection{Furatszerelt alkatrészek beforrasztása}
        
        \subsubsection{Minőségellenőrzés}

        \subsubsection{A PIC felprogramozása}

    \subsection{Ellenőrző kérdések}


\end{document}